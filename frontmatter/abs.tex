\newpage
\TOCadd{Abstract}

\noindent \textbf{Supervisory Committee}
\tpbreak
\panel

\begin{center}
\textbf{ABSTRACT}
\end{center}

Several anomalies within the Standard Model of particle physics exist which could be explained by means of light new physics.
These may be associated with muons, with the gyromagnetic ratio of the muon being different than theory at the level of $3.4\sigma$, and with the $7\sigma$ muonic Hydrogen proton radius extraction.
New, current, and previous experiments may be able to place stringent limits on the existence of a new scalar force, which couples with leptons proportional to their mass.
We investigate the sensitivity to the parameter space of the muon decay experiment \mueee, the kaon decay experiments NA48/2 and NA62, and the experiments at asymmetric electron-positron colliders, \babar, \belle, \belletwo.
Using Monte Carlo techniques to generate events for processes corresponding to each experiment, we find that these experiments could be sensitive to couplings to muons down to $10^{-5}$, and over a wide mass range from $1.1\textrm{MeV} - 3.5\textrm{GeV}$.
