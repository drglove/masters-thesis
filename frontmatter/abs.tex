\newpage
\TOCadd{Abstract}

\noindent \textbf{Supervisory Committee}
\tpbreak
\panel

\begin{center}
\textbf{ABSTRACT}
\end{center}

There are several anomalies within the Standard Model of particle physics that may be explained by means of light new physics.
These may be associated with muons, with the gyromagnetic ratio of the muon being different than theory at the level of $3.4\sigma$, and with the $7\sigma$ muonic Hydrogen proton radius extraction.
Previous, current, and new experiments may be able to place stringent limits on the existence of a new scalar force, with a coupling to leptons proportional to their mass.
We investigate the sensitivity to the parameter space of the muon decay experiment \mueee, the kaon decay experiments NA48/2 and NA62, and the experiments at asymmetric electron-positron colliders, \babar, \belle, \belletwo.
Using Monte Carlo techniques to generate events for processes corresponding to each experiment, we find that these experiments could be sensitive to muonic couplings down to $10^{-5}$, and over a wide mass range of $10\textrm{MeV} - 3.5\textrm{GeV}$, fully covering the parameter space relevant for explanations of these anomalies.
The possibility exists to later extend to masses up to $10~\textrm{GeV}$.
