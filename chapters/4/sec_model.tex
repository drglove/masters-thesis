\section{Models}

Keeping in mind that the discrepancy of the proton-charge radius is about $4\%$, we must be interested in small couplings \cite{Carlson:2015jba}.
Our primary model of interest is the addition of a scalar, $\phi$, with a weak, asymmetric coupling to leptons violating lepton universality.
We are also interested in the dark photon model, as it provides a similar low mass mediator that is weakly coupled to leptons.
In the dark photon model, this is instead a vector as opposed to a scalar.
 \improvement{Add more general model requirements}

\subsection{Extra Scalar}
The Lagrangian we use for the addition of a scalar ends up looking like the Higgs' Lagrangian after electroweak symmetry breaking, which couples to leptons.
We add a real scalar with Yukawa couplings to the three generations of mass, and a standard kinetic term, where the coupling strength is proportional to the mass of the lepton.
This obviously requires picking a mass scale for the coupling, one of two free parameters of the theory.
The other free parameter is, as usual, the mass of the new particle.
Choosing the coupling to scale proportionally to the mass may give a natural way to couple the new scalar more strongly to the $\mu$ than the $e$.
The full Lagrangian is given below in, equation \ref{eqn:scalar_lagrangian}: \unsure{below vs in?} 
\begin{equation}
\label{eqn:scalar_lagrangian}
\Lagr = \frac{1}{2}\partial_\mu \phi \partial^\mu \phi - \frac{1}{2} m_\phi^2 \phi^2 + \sum_{\ell=e,\mu,\tau} g_{\phi \ell}~\bar{\ell}^\lambda~\phi~\ell^\lambda + g_{\phi p}~\bar{p}~\phi~p + \Lagr_{\textrm{SM}}
% g_{\phi e}\bar{e}^\lambda e^\lambda \phi
\end{equation}\unsure{Sum over leptons?} 
where $\ell$ takes on the lepton fields, the couplings in $g_{\phi \ell}$ are the couplings between the scalar and the corresponding lepton, $p$ is the proton field, $g_{\phi p}$ is the coupling between the scalar and the proton, $\Lagr_\textrm{SM}$ is the Standard Model Lagrangian, $\lambda$ indicates spinor indices, and $g_{\phi \ell} \propto m_\ell$.
We usually fix the coupling to the $\mu$ in this thesis, however we will also fix the coupling to the $\tau$ in one portion so it is convenient to write the couplings as:
\begin{equation*}
\label{eqn:coupling_mass}
g_{\phi \ell} = \left(g_{\phi e}, g_{\phi \mu}, g_{\phi \tau}\right) = g_{\phi \mu} \left( \frac{m_e}{m_\mu}, 1, \frac{m_\tau}{m_\mu} \right) = g_{\phi \tau} \left( \frac{m_e}{m_\tau}, \frac{m_\mu}{m_\tau}, 1 \right)
\end{equation*}

Of most importance is the relative magnitude of couplings.
We would like a weak coupling to electrons such that we do not disturb the proton radius as extracted in the electron-proton scattering experiments, and measurement of the atomic energy levels of electronic Hydrogen.
Indeed the point here is to have a stronger coupling to the muon with the scalar, than that of the electron with the scalar.
Taking the couplings to be Higgs-like gives the size of any physical effect to be approximately some power of $(m_\mu/m_e) \approx 210$ times larger for the muon than to the electron.
Going to higher mass scales allows easier probing with this model.
If one is able to access the $\tau$ lepton, the coupling becomes large enough for a deviation to be seen; in particular, a resonance at the mass of the scalar mediator.

It is also possible to allow for a pseudo-scalar coupling to be introduced, by introducing another set of couplings to the pseudo-scalar field and sandwiching an $i\gamma^5$ between spinors.
While it is outside the scope of this thesis to examine this, we will note here that the scalar and pseudo-scalar have corrections to the $(g-2)_\mu$ with opposing signs.
For small masses of the mediator however, the degree of fine tuning required is large enough that we can ignore the pseudo-scalar case if desired, as the correction from the scalar case alone is small enough to give a desirable $(g-2)_\mu$ correction.
This is given in more detail in \cite{Carlson:2015jba}.\improvement{add plot from Carlson fig. 11}\unsure{does the same thing happen for proton radius?}

\subsection{Dark Photon}
The dark photon model adds a new $U(1)'$ force where the mediator now carries a mass.
Giving the dark photon, known as the $A'$ (or alternatively the $V$), a kinetic mixing with the photon, allows it to couple to charged particles with a strength depending on the mixing strength.
This model has been thoroughly examined in the literature as a potential dark matter candidate, and is a fairly general extension to the SM \cite{Holdom:1985ag}.
It also has potential to solve the $(g-2)_\mu$ problem \cite{Pospelov:2008zw}, however, more recently the parameter space has effectively closed off the dark photon as a solution to $(g-2)_\mu$ \cite{Batley:2015lha}.

The full Lagrangian is given below in equation \ref{eqn:darkphoton_lagrangian}:
\begin{equation}
\label{eqn:darkphoton_lagrangian}
\Lagr = -\frac{1}{4}F'^2_{\mu\nu} + \frac{1}{2}m_{A'}^2 A'^2_\mu - \frac{\epsilon}{2}F'_{\mu\nu} F^{\mu\nu} + \Lagr_{\textrm{SM}}
\end{equation}
Here $F'_{\mu\nu} \equiv \partial_\mu A'_\nu - \partial_\nu A'_\mu$ is the field strength tensor corresponding to the dark photon vector field $A'_\mu$, $F_{\mu\nu}$ is the typical QED electromagnetic field strength tensor, $m_{A'}$ is the corresponding dark photon mass, and $\epsilon$ is the kinetic mixing strength.
The mass term appearing here actually could be inserted from a new Higgs field breaking the $U(1)'$ symmetry, and the QED field strength tensor replaced with the $F^Y_{\mu\nu}$ tensor from the SM $U(1)^Y$ electroweak group.
From a phenomenological point of view, it is sufficient to leave the mass term as it appears for our purposes.

After integrating the kinetic mixing term appearing in the Lagrangian by parts, the coupling to charged leptons (and quarks) becomes apparent as:
\begin{equation}
    \label{eqn:darkphoton_qed_current}
    \Lagr \supset \epsilon A'_\mu J^\mu_{EM}
\end{equation}
where $J^\mu_{EM} \equiv e \overbar{\psi} \gamma^\mu \psi$ is the electromagnetic current.
We then have a coupling between fermions and the dark photon with strength $\epsilon e$, i.e.\ it looks just like the typical QED coupling but suppressed by a factor of $\epsilon$.

This may look like the coupling to each lepton is the same, and it is.
It is not immeditately clear that one can have an assymetric effect across leptons on the charged proton radius, or the magnetic moment.
However in the case where a muon is concerened, the virtual particles may be much more massive.
For this reason, the muon is much more sensitive to undiscovered particles, such as an $A'$, than the electron, regardless of the couplings being the same between muon and electron.\improvement{more detail on coupling}
