\section{Limits on NA48/2 and NA62 from Charged Kaon Decays}
In contrast to the previous section on muon decay, kaon decay is relatively easy to handle and can be done even analytically.
We will still make use of \feyncalc in order to compute the traces we require quickly, and \mathematica to compute the final state integration, although it is not necessary under certain approximations and is genuinely computable by hand.
While NA48/2 and NA62 won't reach a number of kaon decays comparable to the muon decays, we still have $2.0 \times 10^{11}$ $K^+$ decays at NA48/2 and an expected $9.0 \times 10^{12}$ $K^+$ decays at NA62.
These numbers are large enough to put strong bounds on the scalar, but these will not be as strong as those from \mueee.
NA48/2 and NA62 will have many kaons decaying with a forward momentum of $75\textrm{GeV} \pm 1\%$, while in the fiducial volume, so we must be careful to take care that the forward detector can actually see any decays we propose.
Furthermore, the background is more complicated and requires special treatment, which we will talk about now.

\subsection{Background}
