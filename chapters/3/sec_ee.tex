\section{$e^+ e^-$ Collisions at B-factories}
\label{sec:ee_experiment}
The B-factories \babar~\cite{Boutigny:1995ib}, \belle~\cite{Cheng:1995im}, and the future experiment \belletwo~\cite{Abe:2010gxa} provide high intensity collisions while still reaching higher energies than typical meson decays have access to.
These experiments are used to produce many $B$ mesons, by tuning the energy to the $\Upsilon(4S)$ resonance.
The $\Upsilon(4S)$ is a $b\bar{b}$ quark bound state with a large preference to decay to $B$ mesons, having a branching ratio $> 0.96$ \cite{Agashe:2014kda}.
By tuning the energy to this threshold, by selecting an $e^+ e^-$ collider with energies $3.5\textrm{GeV}$ and $8\textrm{GeV}$ respectively yielding a total $\sim 2\sqrt{E_+ E_-} = 10.58\textrm{GeV}$ centre of mass energy, the $\Upsilon(4S)$ is just barely produced and must be at rest in the centre of mass frame.
The \kekb collider provides the high and low energy rings of electrons and positrons, called the HER and the LER, for the \belle experiment.
The \babar detector was based at SLAC.
Note that this energy is asymmetric.
This is to give any decay products a boost down the beam pipe before decaying, extending their lifetimes in the lab frame due to Lorentz time dilation.
It also allows one to more clearly see the vertex displacement from the production of the $\Upsilon$ to its decay point.
To this end, the detector is also built with this forward direction in mind.
This energy just happens to be right above the threshold for a decay into a pair of $B$ mesons, which have interesting properties to study.

For instance, one can measure CP violation with them, as the neutral $B$ mesons can oscillate into their anti-particle.
That is to say that $\bar{B^0} \leftrightarrow B^0$.
This was the primary mission of the \babar experiment.
Over their lifetimes, where \belle ran from 1999-2010, \belle collected $\sim 1000fb^{-1}$ worth of data, and \babar roughly half of this.
The new experiment, \belletwo, will be operating in conjunction with upgrades to the \kekb collider, where it will be rebranded as the \superkekb collider \cite{Akeroyd:2004mj}.
Commissioning for the new collider and detector upgrades begin in 2016.
It is expected that $100$ times the integrated luminosity in $e^+ e^-$ decays will be collected over its lifetime, reaching $\order(10-100)ab^{-1}$.
It will operate at the same centre of mass energy, but with less of a tradeoff to the asymmetry, by using a $4\textrm{GeV}$ positron beam and a $7\textrm{GeV}$ electron beam.

In this work, these B-factories are interesting because they allow access to mediator masses up to $\sim 10\textrm{GeV}$.
Our model will couple leptons to a new scalar particle with a strength proportional to the mass of the lepton.
In the other experiments discussed so far, this limits us to coupling to muons.
However, at this scale, tau leptons are copiously produced with enough room in the phase space to still have associated emission of dark scalars.
For our purposes of exploring the sensitivity reach, we will consider these three experiments to be the same, with a difference in the total integrated luminosity.
We do not simulate the detector performance.
This will be sufficient for this work.
