NA48/2 \cite{Batley:1999fv} and NA62 \cite{Martellotti:2015kna} (also known as NA48/3) are two experiments studying rare kaon decays at the CERN Super Proton Synchrotron (SPS).
\section{Charged Kaon Decays at the Super Proton Synchrotron}

\subsection{NA48/2}
The first experiment of the two finished taking data in 2004, yet limits on new physics can still be done using the data collected on tape.
NA48/2 is based on the upgraded NA48 experiment, and was primarily designed to look for charge-parity (CP) violation in the decays of the charged kaons:

\begin{align}
K^\pm & \rightarrow \pi^+ + \pi^- + \pi^\pm \\
K^\pm & \rightarrow \pi^0 + \pi^0 + \pi^\pm
\end{align}

\subsection{NA62}
On the other hand, NA62 is looking to measure the Cabbibo-Kobayashi-Maskawa (CKM) matrix element $|V_{td}|$ at the level of $10\%$ by measuring the very rare charged kaon decay:
\begin{equation}
K^+ \rightarrow \pi^+ + \nu + \bar{\nu}
\end{equation}
In October 2014, NA62 successfully launched and began taking data.
