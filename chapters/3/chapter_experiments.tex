\startchapter{Experiments}
\label{chapter:experiments}

The intensity frontier may have means to access new light physics, even with extremely weak couplings to the SM.
Many models involving new low mass states have been investigated at these types of experiments, as the energy frontier typically is associated with a lower luminosity.
B-factories have incredibly high collision rates.
\belle, at the \kekb accelerator, has reached a total integrated luminosity of $1043fb^{-1}$, allowing one to probe small deviations from the SM in $e^+ e^-$ collisions.
This also extends to decays, where experiments such as \mueee will examine $10^{15}$ muon decays across the experiment's lifetime.
Kaon decays are also explored at NA48/2 at the SPS at CERN, which lives in an intersection of the energy and intensity frontiers, and has provided some $10^{11}$ kaon decays to study.

Adding a loosely coupled light particle can potentially add new decay modes with branching ratios $\propto \epsilon^2$, with $\epsilon$ being the coupling between the new physics and the SM.
Most branching ratios are limited to be below $10^{-8}$ for new physics, with the mass of the mediators taken to be on the MeV scale.
Limits from current experiments force one to accept either ``dark'' ({\em i.e.}\ weakly coupled) or heavy new physics.
High luminosities and large numbers of decays provide a means to access the light, weakly coupled new physics.

This thesis is concerned with examining the existing and future sensitivity reach one can have with such models, at various experiments.
By accessing different experiments, one can place limits for various ranges of the allowable masses.
We will be looking at three physically different phenomenon across six experiments.
\begin{enumerate}
    \item \mueee is an experiment dedicated to looking at large numbers of muon decays, which allows limits on the mass range from $2 m_e < m_\phi < m_\mu$.
    \item NA48/2, and its successor NA62, provide access to many kaon decays, which will allow placing limits on the mass range from $m_\mu < m_\phi < m_K - m_\mu$.
    \item \babar, \belle, and future experiment \belletwo, use $e^+ e^-$ collisions near centre of mass energy of $10\textrm{GeV}$, considerably extending the reach in terms of the mass of the $\phi$ scalar, $2m_e < m_\phi < E_\textrm{CM} - 2 m_\mu$. Backgrounds can be expected to be better in the region of $m_\phi > 2 m_\mu$.
\end{enumerate}

Our final results will also include constraints from the beam dump experiment E137 conducted in the 1980's \cite{Bjorken:1988as}.
These computations are part of a work in progress \cite{Batell:2015unpub}.
As we will see in this chapter, many of our estimates will fail below $10~\textrm{MeV}$.
E137 is able to access low masses of the scalar and is able to place strong limits below $10~\textrm{MeV}$, and so we can rely on its constraints in this range.
We will not discuss E137 further in this thesis.

\input chapters/3/sec_muon
\input chapters/3/sec_kaon
\input chapters/3/sec_ee
