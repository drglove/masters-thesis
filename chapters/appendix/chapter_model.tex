\startappendix{Model Generation Files}
\label{chapter:modelappendix}
We provide the relevant files for generating a usable \feynrules model for the models used in this thesis.

\section{Scalar Model}
\subsection{\texttt{scalar.fr} Field Definition}
This file defines the scalar field and its parameters.
Note that it does not say anything about the width, which will be automatically computed for all available $1 \rightarrow 2$ processes when generating events.
It also does not contain any information on the allowed interactions, which come in the model file.
\lstinputlisting{feynrules/scalar.fr}

\subsection{\texttt{scalar.m} Lagrangian and Feynman Rules}
This file defines the Lagrangian to be used, including both the freely propagating component and the interaction component.
\feynrules is invoked to produce a computationally usable model.
\lstinputlisting[language=mathematica]{feynrules/scalar.m}


\section{Dark Photon Model}
The same comments regarding the scalar model hold here also.
\subsection{\texttt{darkphoton.fr} Field Definition}
Unlike the scalar model, we must define the $U(1)'$ gauge group here.
\lstinputlisting{feynrules/darkphoton.fr}
\subsection{\texttt{darkphoton.m} Lagrangian and Feynman Rules}
Note that \feynrules has a function \texttt{FS}, which is allows one to use the field strength tensor when formulating a Lagrangian.
\lstinputlisting[language=mathematica]{feynrules/darkphoton.m}
