\startfirstchapter{Introduction}
\label{chapter:introduction}

The Standard Model of particle physics stands as one of our most scrutinized and well tested theories within physics.
Very few discrepancies have been found between the model and reality within the realm of experimental particle physics.
The discovery of the Higgs boson in July 2012 stands as the most recent example, with the measured quantities thus far matching the Standard Model Higgs boson.
However, it must be the case that the theory is incomplete, as we currently do not know how to incorporate the effects of gravity into the model.
In order to make progress, one can look to the few experiments which provide some level of discrepancy.
These could act as a loose string which we can use to unravel the current mysteries of particle physics.

Muons have long since been a source of mystery within the physics community.
Since their discovery in 1936 by Carl Anderson and Seth Neddermyer, in which particles that curved in a magnetic field less sharp than the trajectory of an electron, but sharper than that of a proton, there have been many questions regarding the properties of this particle.
One of the deeper questions of the Standard Model concerns the number of generations of leptons: why is it that we see three generations, the electron, muon, and tau? Within the Standard Model, these only differ by the mass assigned to them.\unsure{Should I start more generally? Dark matter, etc.}

However, while this can be viewed as an aesthetic problem with the model, at least two experimental efforts have revealed discrepancies between the Standard Model muons and reality.
In particular, the muon's spin magnetic moment $g$ appears to be different at the level of $10^{-8}$, which while a small number, represents a $3.4\sigma$ discrepancy between experiment and theory \cite{2007PhLB..649..173H}.

Additionally, one can perform experiments on \emph{muonic Hydrogen} ($\mu\textrm{H}$) in which the electron orbiting a proton is replaced with a muon.
In both cases of electronic Hydrogen (H) and $\mu\textrm{H}$, the proton charge radius, which will be better defined later in the thesis, can be extracted.
Peculiarly, the two results differ by a significant amount \improvement{how much?}\improvement{reference?} and stand at a $7\sigma$ discrepancy \cite{Carlson:2015jba}.
Regardless of whether we use H or $\mu\textrm{H}$, the extraction of the proton charge radius should remain unchanged and this leads to an exciting avenue to search for new physics, and could perhaps give some hints to a new physics role of the muon in such a system.

This thesis contains six core components.
The first of these is this short introduction to the problem and motivation.
Here we provide a small summary of the contents of each chapter that has not already been covered.

\textbf{Chapter 2} discusses the physical motivations for this project in detail.
This mainly focuses on the apparent anomalies that lie within the Standard Model with emphasis on the muon, and lepton flavour violation.
A delve into the proton radius problem will be the primary motivation, but we will also discuss the $(g-2)_\mu$ discrepancy briefly.
This chapter will stand as the summary of the experimental hints that push us to pursue this project.
From this, we will see that a candidate for solving these problems is the addition of a new scalar force.
This candidate particle will then be the focus of the rest of the thesis, and limits of detection will be the final result.

\textbf{Chapter 3} will examine each experiment in enough detail to place sensible limits on the scalar particle.
These experiments will span the mass range of a few $\textrm{MeV}$, up to $10\textrm{GeV}$.
To do this, three classes of experiments will be examined: muon decay, kaon decay, and $e^+ e^-$ collisions.
Muon decay will be studied in the context of the upcoming \mueee experiment, which provides a large number of muon decays to work with.
Kaon decay at the finished experiment NA48/2, and the upcoming NA62 experiment, will provide access to mediator masses above the muon mass.
Finally, $e^+ e^-$ collisions at a centre of mass energy of $10.58\textrm{GeV}$, as analyzed by the B-factories \babar, \belle, and \belleii will provide the greatest coverage of masses.

\textbf{Chapter 4} provides the prescription of tools that we use, and the analytical methods in which we extract our limits.
The tools include the Monte Carlo generator \madgraph, and other utility tools to interface with the generator.
These are used to generate large numbers of signal and background events, in which to study the sensitivity to our model, as finding an analytic expression for the decay rates and cross sections can be difficult.
Our limit extraction procedure is also covered here, where we precisely define what we claim an experiments upper-limit sensitivity is.

\textbf{Chapter 5} contains the results of this work, which are the limits on the experiments listed above.
Each process analyzed is documented here in detail, and the resulting limit is shown.

\textbf{Chapter 6} will conclude the thesis, with a summary of the work completed and results.
There will also be some brief comments on future work in this area.

\redo{Finish intro later}

