\startfirstchapter{Introduction}
\label{chapter:introduction}

The Standard Model of particle physics stands as one of our most scrutinized and well tested theories within physics. Very few discrepancies have been found between the model and reality within the realm of experimental particle physics. The discovery of the Higgs boson in July 2012 stands as the most recent example, with the measured quantities thus far matching the Standard Model Higgs boson. However, it must be the case that the theory is incomplete, as we currently do not know how to incorporate the effects of gravity into the model. In order to make progress, one can look to the few experiments which provide some level of discrepancy. These could act as a loose string which we can use to unravel the current mysteries of particle physics.

Muons have long since been a source of mystery within the physics community. Since their discovery in 1936 by Carl Anderson and Seth Neddermyer, in which particles that curved in a magnetic field less sharp than the trajectory of an electron, but sharper than that of a proton, there have been many questions regarding the properties of this particle. One of the deeper questions of the Standard Model concerns the number of generations of leptons: why is it that we see three generations, the electron, muon, and tau? Within the Standard Model, these only differ by the mass assigned to them.\unsure{Should I start more generally? Dark matter, etc.}

However, while this can be viewed as an aesthetic problem with the model, at least two experimental efforts have revealed discrepancies between the Standard Model muons and reality. In particular, the muon's spin magnetic moment $g$ appears to be different at the level of $10^{-8}$, which while a small number, represents a $3.4\sigma$ discrepancy between experiment and theory \cite{2007PhLB..649..173H}.

Additionally, one can perform experiments on \emph{muonic Hydrogen} ($\mu\textrm{H}$) in which the electron orbiting a proton is replaced with a muon. In both cases of electric Hydrogen (H) and $\mu\textrm{H}$, the proton charge radius, which will be better defined later in the thesis, can be extracted. Peculiarly, the two results differ by a significant amount \improvement{how much?}\improvement{reference?} and stand at a $7\sigma$ discrepancy \cite{Carlson:2015jba}. Regardless of whether we use H or $\mu\textrm{H}$, the extraction of the proton charge radius should remain unchanged and this leads to an exciting avenue to search for new physics, and could perhaps give some hints to a new physics role of the muon in such a system.

\redo{Finish intro later}

