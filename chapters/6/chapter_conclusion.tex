\startchapter{Conclusions}
\label{concl}

My first rule for this chapter is to avoid finishing it with a section talking about future work. It may seem logical, yet it also appears to give a list of all items which remain undone! It is not the best way psychologically.

This chapter should contain a mirror of the introduction, where a summary of the \textit{extraordinary} new results and their wonderful attributes should be stated first, followed by an executive summary of how this new solution was arrived at. Consider the practical fact that this chapter will be read quickly at the beginning of a review (thus it needs to provide a strong impact) and then again in depth at the very end, perhaps a few days after the details of the previous 3 chapters have been somehow forgotten. Reinforcement of the positive is the key strategy here, without of course blowing hot air.

One other consideration is that some people like to join the chapter containing the analysis with the only with conclusions. This can indeed work very well in certain topics.

Finally, the conclusions do not appear only in this chapter. This sample mini thesis lacks a feature which I regard as absolutely necessary, namely a short paragraph at the end of each chapter giving a brief summary of what was presented together with a one sentence preview as to what might expect the connection to be with the next chapter(s). You are writing a story, the \textit{story of your wonderful research work}. A story needs a line connecting all its parts and you are responsible for these linkages.
