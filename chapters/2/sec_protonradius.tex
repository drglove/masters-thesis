\section{Proton Radius Puzzle}
As has been mentioned a few times already in this thesis, if one does experiments to extract the proton radius using $\mu\textrm{H}$, a drastically different result is pulled out compared to simply $H$.
In this section we will summarize how this is extracted, and potential solutions, including the scalar we put forward, following mostly \cite{Carlson:2015jba}.

First, there are multiple methods which are currently used to extract the proton radius from electronic Hydrogen, H.
These typically have large error bars, but do agree with each other, to yield a combined uncertainty of $\sim 0.6\%$.
The experiments to extract the proton radius in H are either done using electron-proton scattering, or by measuring energy levels of the Hydrogen system.

For non-relativistic scattering experiments, the differential cross section is given by

\begin{equation}
    \frac{d\sigma}{d\Omega} = \left(\frac{\alpha}{4E\sin^2(\theta/2)}\right)^2 \times \left(G(Q^2)\right)^2
\end{equation}

\noindent with the first term being the typical scattering of an electron off of a point-like proton, and the second term correcting for the finite size of the proton.
This second term is named a form factor and depends on the momentum transfer.
Moving to a relativistic system, as is required for any electron-proton scattering experiment, we find a similar expression

\begin{equation}
    \frac{d\sigma}{d\Omega} = \frac{4 \alpha^2 \cos^2(\theta/2)}{Q^4} \frac{E'^3}{E} \times \frac{1}{1+\tau}\left(G_E^2(Q^2) + \frac{\tau}{\epsilon}G_M^2(Q^2)\right)
\end{equation}

\noindent with $\tau = Q^2/4M^2$ and $1/\epsilon = 1 + 2(1+\tau)\tan^2(\theta/2)$, with $E'$ and $E$ being the outgoing and incoming energies of the electron in the lab frame.
The first term is now the cross section when scattering an unpolarized spin-1/2 electron off of a pointlike spin-0 target.
Now the form factors are separated into the electric and magnetic components.

To define the proton radius, we look to the non-relativistic system and find that $G(Q^2) = 1 - \frac{1}{6}\langle r^2 \rangle Q^2 + \cdots$, which has dependence on the average squared radius.
Promoting this to a definition, we define the charge radius to be

\begin{equation}
    R_E^2 \equiv -6 \frac{d G_E}{d Q^2}\rvert_{Q^2=0}
\end{equation}

\noindent Experiments looking to extract the proton charge radius using $e-p$ scattering must then measure this form factor.

One can also measure the Lamb shift, the difference between the $2S_{1/2}$ and $2P_{1/2}$ levels.
The correction to the splittings is given by

\begin{equation}
    \Delta E = a \textrm{Ry} + b R_E^2
\end{equation}

\noindent for known coefficients a and b. If one measures two splittings at a time, both the Rydberg constant, and the proton charge radius can be extracted simultaneously, and this is precisely what is done.
The overall result is $R_E = 0.8775(51)\textrm{fm}$ when everything is combined from CODATA~\cite{Mohr:2012tt}.

Similarly, a group performed the same extraction by measuring the Lamb shift of $\mu\textrm{H}$~\cite{Pohl:2010zza,Antognini:1900ns} and found $R_E = 0.84087(39)\textrm{fm}$ after two trials.
This is the $7\sigma$ discrepancy between the muonic Hydrogen and electric Hydrogen.
The experiment involved work at the Paul Scherrer institute with the Charge Radius Experiment with Muonic Atoms (CREMA) experiment.
A beam of muons are slowed and sent into a target of hydrogen, where they are captured in many high energy states, proceeding to cascade down to the $1S$ or $2S$ state.
A laser is then shined on the $\mu\textrm{H}$ to bump the $2S$ states up to the $2P$ states, which readily decay to the $1S$ state giving off a characteristic X-ray.
The frequency the laser is tuned at will then give the splitting and the proton radius can be read off, provided one has an X-ray detector that can signal when you have tuned the laser correctly.
The exact details of determining the splitting here an not important to our discussion, so this description shall suffice.

The conclusion is that there are unknown/unexpected QCD corrections that can provide the necessary correction, there is new physics that violates lepton universality, or that there is some problem with the extraction of $R_E$ from the collection of electronic measurements.
In this thesis, we focus on the possibility of new physics.
It's useful to note that there exist points in the parameter space which can solve both the proton radius problem and the discrepancy of $(g-2)_\mu$.
We can use this as a constraint during model building.
This is the major motivator we have when approaching this thesis.
What sort of couplings are permitted by experiment for a certain class of scalar particles?
