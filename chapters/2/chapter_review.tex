\startchapter{Brief Review of Interesting Muonic Features}
\label{chapter:review}

The main motivation for the thesis is that there exist experimental anomalies that have yet to be explained, or disappear as statistically insignificant.
This chapter is dedicated to investigating these anomalies in detail.
There are two main experimental results that motivate light new physics for our purposes.

\begin{enumerate}
    \item The proton radius puzzle, in which the proton radius is extracted from $\mu\textrm{H}$.
    \item $(g-2)$ for the muon, which may be explained by loops of light mediators.
\end{enumerate}

We will also discuss an alternative theory to the scalar, a vector known as the dark photon, that we put forward to test and may also solve the problems above.
Both of these concern the muon, which may provide a hint as of where to look for new interesting phenomenon not yet explained by the SM.
Within the SM, there is no lepton flavour violation, except when one adds neutrino masses, at which point charged lepton flavour violation becomes allowed, but is suppressed by the incredibly small difference of the neutrino's mass squared, normalized on the weak scale coupling $G_F$.
Another SM particle which breaks lepton universality between the electron and muon is the SM Higgs.
The exchange of a Higgs boson at low energy gives rise to the four fermion interaction term

\begin{equation}
\label{eqn:higgs_fourfermi}
\frac{1}{m_h^2} \frac{m_i m_j}{v^2} \bar{\ell_i} \ell_i \bar{\ell_j} \ell_j
\end{equation}

\noindent with $m_h$ being the Higgs mass, and $v$ the Higgs vacuum expectation value (vev).
This interaction is suppressed due to the small masses of the leptons, the large mass of the Higgs, and the even larger vev.
Any possible effect from equation \ref{eqn:higgs_fourfermi} is tiny, making it irrelevant for the phenomenological signatures we discuss.
This tells us that, if we take into account the masses of the charged leptons correctly, then physics should be no different when we replace an electron with a muon.
Of course this has limits, since the muon can decay to an electron and anti-muon neutrino, and the electron has no decay modes we are aware of, but this is more a question of allowable kinematics.
To have fundamentally different interactions between a muon and other SM particles would be a smoking gun for new physics.

\input chapters/2/sec_protonradius
\input chapters/2/sec_gmuon
